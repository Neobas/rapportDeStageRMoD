% section 3 : Activités du stage
\section{Activités du stage}
Dans cette partie, je présenterai le fruit de mon travail au sein de l'équipe durant le stage.

\subsection{Appel à la communauté Pharo}
Avant de commencer les modifications, j'ai effectué un sondage pour connaître l'avis de la communauté au sujet du comportement que devrait adopter Pharo dans un environnement en lecture seule, qui a reçu 25 réponses en 2 semaines environ. Après analyse des résultats, j'ai défini le comportement qui devait être implémenté.

\subsection{Amélioration de la gestion des fichiers}

\subsubsection{Fichiers de log}
Le sondage a révélé que 68\% des sondés désirent que le log soit écrit dans un emplacement par défaut défini par le système (/var/log/ pour Linux par exemple), 8\% désirent que le log soit juste imprimé dans le terminal et 4\% désirent qu'il soit stocké en mémoire, mais avec la possibilité de pouvoir choisir l'endroit où sera écrit le log. J'ai donc élaboré un système de log basé sur les design patterns Singleton, Factory et Strategy. Ce système est configurable depuis la ligne de commande, et repose sur des stratégies définissants le comportement que doit adopter Pharo pour logger. J'ai défini 5 stratégies par défaut, en laissant à l'utilisateur la possibilité d'en créer de nouvelles :

\begin{description}
	\item[local :] La stratégie par défaut de Pharo. Elle écrit le log dans le répertoire courant.
	\item[usual :] La stratégie la plus demandée. Elle écrit le log dans l'emplacement par défaut prévu par le système.
	\item[explicit :] La stratégie offrant le plus de liberté. Elle permet à l'utilisateur de définir l'emplacement du fichier de log.
	\item[stdout :] La stratégie de sûreté. Elle imprime le log sur la sortie standard, ce qui ne sauvegarde rien et assure donc que le log sera bien inscrit quelque part.
	\item[memory :] La stratégie de test. Elle simule le comportement d'un fichier en écrivant dans une chaîne de caractères. Elle est surtout utile pour tester l'écriture, mais peut être utilisée partout.
\end{description}

