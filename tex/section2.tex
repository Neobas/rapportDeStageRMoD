% section 2 : activités
\section{Activités du stage}
\subsection{Analyse}
Cette section aborde les informations récoltées pour définir les problèmes à résoudre et les corrections à apporter.
\subsubsection{Les tests sur le système}
Pour connaître les problèmes liés au déploiement de Pharo dans un environnement en lecture seule, j'ai élaboré une liste de tests que j'ai effectués en conditions "normales", c'est-à-dire dans un répertoire où l'écriture est possible, et dans un répertoire en lecture seule. Je passerai en revue ces tests ci-dessous, en spécifiant le comportement observé dans chacun des cas, et les classerai en 3 catégories :
\begin{description}
	\item[Fonctionnement normal :] La fonctionnalité a le comportement attendu, il n'y a rien à corriger.
	\item[Fonctionnement altéré :] La fonctionnalité a un comportement inattendu, mais ne renvoie pas forcément d'erreur.
	\item[Fonctionnement impossible :] La fonctionnalité est inutilisable en l'état, et renvoie systématiquement une erreur.
\end{description}

\paragraph{Le système de log :}
Le log est un mécanisme permettant aux développeurs d'identifier des problèmes lors de l'exécution d'un programme en écrivant la trace de celle-ci dans un fichier. Pour tester ce système, j'ai exécuté la commande suivante :
\begin{verbatim}
Smalltalk logDuring: [ :stream |
                        stream nextPutAll: 'Hello log'; cr ]
\end{verbatim}
Cette commande envoie le message \verb$logDuring:$ à la classe \verb$Smalltalk$, ce qui a pour effet d'exécuter le bloc placé en paramètre en lui donnant accès à un stream vers le fichier de log. Ici, le bloc est le suivant :

\begin{verbatim}
[ :stream | stream nextPutAll: 'Hello log'; cr]
\end{verbatim}

Ce dernier prend un paramètre nommé \verb$stream$ et lui envoie le message \verb$nextPutAll:$ , ce qui écrit dans le stream la chaîne de caractères \verb$'Hello log'$. Puis le message \verb$cr$ est envoyé en cascade pour effectuer un retour chariot. Le résultat attendu est l'apparition du message "Hello log" dans le fichier de log.

Dans des conditions "classiques", le message est correctement écrit dans le fichier.

Dans le cas du système Read-Only, il ne se passe rien : aucune erreur n'apparaît, ce qui est un bon point, cependant, en examinant le fichier de log on observe que le message n'est pas écrit. J'ai donc classé ce problème dans la catégorie "Fonctionnement altéré".

\paragraph{Lancement des tests (Test Runner) :}
Pharo, c'est 6499 classes, 86479 méthodes et 13395 tests, ce qui représente beaucoup de travail si on souhaite lancer tous les tests du système. C'est là que le Test Runner trouve tout son intérêt : Ce programme inclus dans Pharo permet de voir et de lancer tous les tests existants avec un interface clair et simple d'utilisation. Son fonctionnement est très intuitif : on sélectionne le ou les paquets à tester dans la colonne de gauche, puis les classes de test dans la colonne de droite. Enfin, on clique sur le bouton correspondant à ce qui nous intéresse, lancement des tests ou couverture du code. On peut ensuite consulter le résultat dans la partie droite de la fenêtre qui indique l'issue des tests lancés avec un code couleur:
\begin{description}
	\item[le fond est vert :] les tests ont tous passé, il n'y a pas de problème.
	\item[le fond est jaune :] il y a des erreurs non-bloquantes, il est conseillé de les corriger.
	\item[le fond est rouge :] il y a des erreurs bloquantes, comme une exception lancée par exemple, il faut corriger au plus vite.
\end{description}

Pour vérifier le fonctionnement du Test Runner, je l'ai ouvert pour lancer tous les tests. Le résultat est bon : il fonctionne aussi bien dans des conditions "normales" qu'en Read-Only, seul le nombre de tests réussis change :
\begin{itemize}
	\item en conditions "normales" : sur 13671 tests lancés, 13599 passent, 26 sont iqnorés, 57 déclenchent une erreur prévue, 7 échouent et 8 erreurs ont étés levées.
	\item en conditions read-only : sur 12620 tests lancés, 11575 passent, 16 sont iqnorés, 87 déclenchent une erreur prévue, 28 échouent et 930 erreurs ont étés levées.
\end{itemize}

Le résultat des tests indique que plusieurs systèmes sont défaillants, mais le Test Runner fonctionne sans problème, je l'ai donc classé dans "Fonctionnement normal".

\paragraph{Écriture de code (Nautilus) :}
Pour écrire du code dans Pharo, on utilise Nautilus, le navigateur du système. Il donne accès aux paquets, classes et méthodes, et permet la modification ainsi que la création de ces derniers. Bien que Nautilus soit amené à être remplacé par Calypso dans Pharo 7, je l'ai tout de même testé pour être complet.

Mon test consistait en la création d'une classe et d'une méthode, et la modification de ces dernières. En conditions "normales", il n'y a aucun problème. En read-only, la création de la classe comme de la méthode provoque l'apparition d'erreurs, mais la création se fait tout de même. La modification d'une méthode fonctionne, mais celle d'une classe freeze l'image.

Il est possible de coder, mais avec des erreurs, je l'ai donc classé dans "Fonctionnement altéré".

\paragraph{La sauvegarde (Monticello) :}
Monticello est un système de contrôle de version (VCS) comme git, c'est-à-dire qu'il permet de sauvegarder les modifications apportées à un moment donné sous forme de commits, permettant ainsi de pouvoir revenir dans le passé via ces derniers. En plus de permettre une sauvegarde en local, Monticello propose de sauvegarder les modifications dans un repository en ligne, via un service comme Smalltalkhub. Monticello est amené à être remplacé dans Pharo 7 par Calypso, un VCS plus centré sur git avec une gestion par repository plutôt que par paquet pour Monticello.

Le test consistait à enregistrer une modification et à charger un paquet dans Monticello. Si en conditions "normales" la manipulation ne pose aucun problème, il est impossible de sauvegarder en read-only, que ce soit en local ou en ligne. La seule option possible est d'enregistrer dans les fichiers temporaires, ce qui ne permet donc pas de faire une vraie sauvegarde. De plus, tenter un commit renvoit une erreur. Pour le chargement, le résultat est plus radical, car ce dernier provoque un freeze de l'image. J'ai donc classé Monticello dans la catégorie "Fonctionnement impossible".

\paragraph{Le playground :}
Le playground est comme son nom l'indique un bac à sable où on exécute du code librement. Servant d'invite de commande, c'est l'un des points d'entrée du système, permettant de lancer un programme, d'imprimer le résultat d'une évaluation et d'analyser la composition d'un objet (un des avantages du système "vivant"), entre autres. Il dispose de plus d'un système d'historique permettant de récupérer le contenu d'une session précédente qui a été fermée par mégarde par exemple.
Mon test visait à tester ce système d'historique en écrivant quelque chose dans le playground avant de le fermer et de tenter dans un autre de récupérer le code écrit. Si cela ne pose aucun problème en conditions "normales", la moindre modification entrée dans le playground (ajout d'un caractère par exemple) déclenchera une erreur non-bloquante, ce qui entrave vraiment l'utilisation du playground, étant obligé d'entrer un caractère et de fermer l'erreur en boucle. De plus, l'historique est indisponible, ce qui place donc le playground dans la catégorie "Fonctionnement impossible".

\subsubsection{Enquête auprès de la communauté}
Avant de commencer à corriger les problèmes identifiés, j'ai fait une enquête (en anglais) auprès de la communauté de Pharo, intitulée "How Pharo should act in a read-only environment ?" (Comment Pharo devrait se comporter dans un environnement en lecture seule ?), pour avoir leur avis sur le comportement que Pharo devrait adopter au terme des réparations. Pour chaque question, plusieurs réponses étaient proposées, tout en laissant aux sondés la possibilité de donner une réponse différente si les propositions ne leurs convenaient pas. Ci-dessous sont listées les questions avec la répartition des réponses et une analyse de celles-ci.

\paragraph{Est-ce que les tests qui écrivent dans un fichier devraient être lancés ?}
\subparagraph{Réponses balisées :}
\begin{itemize}
	\item 24\% des sondés se moquent de savoir si les tests sont lancés ou non, car ces derniers n'ont pas été écrit dans le but d'être lancé en lecture seule.
	\item 20\% des sondés souhaitent que les tests soient lancés, mais que ces derniers doivent échouer si le système est en lecture seule.
	\item 12\% des sondés souhaitent que les tests soient lancés et réussissent que ce soit en conditions "normales" ou en lecture seule.
	\item 8\% des sondés ne souhaitent pas que les tests puissent être lancés en lecture seule.
	\item 8\% des sondés ne se prononcent pas.
\end{itemize}

\subparagraph{Réponses libres :}
\begin{itemize}
	\item Certains tests peuvent être modifiés pour fonctionner dans un environnement en lecture seule, mais pas tous. Une solution pourrait être de diviser les tests en 2 catégories : les "vrais" tests unitaires et les tests "pas vraiment" unitaires.
	\item Ces tests devraient être lancés en mémoire ou être ignorés.
	\item Un mock devrait être généré automatiquement.
	\item Pourquoi tester dans un environnement en lecture seule ? Les tests suggèrent du débug, donc de l'écriture de code, donc de l'écriture tout court.
	\item Si un système est fait pour être lancé en lecture seule, il devrait être possible de marquer (avec un pragma par exemple) les tests qui ne devraient pas être lancés et ceux qui devraient l'être (pour permettre de tester le mode read-only).
	\item Il faudrait ajouter une extension pour les tests écrivant dans un fichier et la désactiver quand le système est en lecture seule.
	\item J'aimerai que tous les tests réussissent, mais attendre des tests qu'ils échouent semble plus raisonnable.
\end{itemize}

\subparagraph{Analyse :}
L'idée générale est que les tests devraient pouvoir être lancés, mais qu'ils devraient soit échouer, soit être ignorés. En soit la première option est déjà implémentée, car les tests échouent inévitablement si l'écriture échoue. La catégorisation est une idée intéressante, mais qui demande un travail plus approfondi pour trouver un système qui soit simple d'utilisation tout en assurant que le système ne sera pas trop impacté.

\paragraph{Devrait-il être possible de créer des classes et/ou des méthodes ?}
\subparagraph{Réponses balisées :}
\begin{itemize}
	\item 60\% des sondés souhaitent être capable d'écrire du code sans être confronté à une erreur.
	\item 16\% des sondés considèrent que le système ne devrait pas accepter de changements si il est en lecture seule.
	\item 8\% des sondés souhaite pouvoir écrire du code, quitte à ce que le système lève des exceptions.
	\item 4\% des sondés ne se prononcent pas.
\end{itemize}

\subparagraph{Réponses libres :}
\begin{itemize}
	\item Il devrait être possible de customiser le système avec un script lancé au démarrage pouvant ajouter des classes, etc. Il n'est pas nécessaire de sauvegarder quoi que ce soit.
	\item Plutôt non (car le système est en lecture seule), ou d'une façon non persistante.
	\item Oui, mais uniquement dans la mémoire.
	\item Nous devrions pouvoir le faire, mais sans sauvegarder.
\end{itemize}

\subparagraph{Analyse :}
La majorité des sondés désire pouvoir coder sans problème, tandis qu'une partie un peu plus petite est prête à abandonner la sauvegarde pour pouvoir écrire du code, supposant que lancer Pharo en lecture seule implique de ne pouvoir qu'exécuter du code.

\paragraph{Devrait-il être possible de sauvegarder les changements dans Monticello ?}
\subparagraph{Réponses balisées :}
\begin{itemize}
	\item 40\% des sondés souhaitent que Monticello abandonne son cache de paquet posant problème pour pouvoir l'utiliser .
	\item 24\% des sondés ne souhaitent pas pouvoir sauvegarder avec Monticello si le système est en lecture seule.
	\item 12\% des sondés ne se prononcent pas.
	\item 8\% des sondés souhaitent que Monticello écrive son cache dans une zone inscriptible pour qu'il puisse fonctionner normalement.
\end{itemize}
%\subsubsection{Problématique}

%\subsection{Conception d'une solution de logging}
%\subsubsection{Architecture}

%\subsubsection{Fonctionnalités}

%\subsubsection{Résolution du problème}

%\subsection{Intégration au système}
%\subsubsection{Modifications apportées}

%\subsubsection{Respect de la rétrocompatibilité}