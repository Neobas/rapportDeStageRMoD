% Section 1 : Contexte du stage
	\section{Contexte du stage}
	\subsection{L'entreprise}
	\subsubsection{Inria}
	L'institut National de Recherche en Informatique et en Automatique (Inria) est un institut français de recherche en mathématiques et informatique, créé le 3 janvier 1967, ayant pour objectif la mise en réseau des compétences et talents de l'ensemble du dispositif de recherche français et international dans ses domaines de compétence. Il dispose de huit centres de recherches autonomes répartis sur tout le territoire français :
	
	\begin{itemize}
		\item Bordeaux - Sud-Ouest
		\item Grenoble - Rhône-Alpes
		\item Lille - Nord Europe
		\item Nancy - Grand Est
		\item Paris - Rocquencourt
		\item Rennes - Bretagne Atlantique
		\item Saclay - Île-de-France
		\item Sophia Antipolis - Méditerranée
	\end{itemize}

	\subsubsection{RMoD}
	RMoD est une équipe de recherche de l'Inria basée dans le centre de Lille - Nord Europe, et dirigée par Stéphane Ducasse. Elle a principalement deux objectifs : le refactoring de grands systèmes et la création de langages de programmation dynamiques et réflexifs.
	
	\paragraph{}
	Dans l'optique de refactoring de grands systèmes, l'équipe propose des outils d'analyse pour comprendre et restructurer les logiciels. Cette partie est assurée par Synectique, une entreprise créée dans ce but.
	
	\paragraph{}
	Dans l'optique de création de langages de programmation, RMoD développe le langage Pharo, un langage dynamique, réflexif et purement objet basé sur Smalltalk.

	\newpage{}
	\subsection{Technologies employées}
	\subsubsection{Pharo}
	Pharo est un langage de programmation open source créé en 2008 par l'équipe RMoD, issu d'un fork de Squeak. Il est basé sur une machine virtuelle (VM) écrite en grande partie en Pharo, ce qui lui la capacité d'être multiplateforme (Mac OS X, Windows, Linux, iOS, Android). Sa politique impose que ses contributeurs publient leur code sous licence MIT.

	\paragraph{}
	Basé sur Smalltalk, Pharo en possède les principales caractéristiques:
	\begin{itemize}
		\item Le système est réflexif.
		\item Le typage est dynamique.
		\item Pharo est implémenté suivant le paradygme objet
		\item L'héritage est simple.
		\item La gestion de la mémoire se fait automatique au moyen d'un garbage collector (GC)
	\end{itemize}

	\paragraph{}
	Détail intéressant : la syntaxe de Pharo tient sur une carte postale.

	\paragraph{}
	L'un des principaux intérêts de Pharo est qu'il n'est pas nécessaire de recompiler tout le code en cas de modification. Il est par exemple possible de coder au sein du débugger : On écrit un test, on le lance, une erreur apparait (le message envoyé n'existe pas), on peut alors ouvrir le débugger, implémenter la méthode correspondante, compiler à la volée et reprendre le cours de l'exécution comme si rien ne s'était produit.

	\paragraph{}
	Pharo est actuellement disponible dans sa version 6.0, sortie le 6 juin 2017.

	\subsubsection{git}
	Git est un logiciel libre de gestion de versions décentralisé créé par Linus Torvalds, et distribué sous la licence publique générale GNU (GPL-2.0). Il est particulièrement utile pour gérer les modifications apportées à un projet grâce à un système de commits et de branches, surtout si il est couplé à un service en ligne capable de le gérer (par exemple github, pour ne citer que le plus connu).

	\subsubsection{LaTeX}
	LaTeX est un langage de composition de documents créé par Leslie Lamport en 1983. Il sert de surcouche destinée à simplifier l'utilisation du langage TeX de Donald Knuth. Il est depuis 1993 maintenu par le LaTeX3 Project team. Utilisant un système de macros, il permet de créer des documents ayant une mise en page se voulant élégante, et est devenu la méthode privilégiée des scientiques pour rédiger des documents de taille moyenne et longue, comme une thèse ou un livre. Ce document est rédigé en LaTeX.