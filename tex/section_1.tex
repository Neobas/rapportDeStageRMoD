% Section 1 : Contexte du stage
	\section{Contexte du stage}
	\subsection{L'entreprise}
	\subsubsection{Inria}
	L'institut National de Recherche en Informatique et en Automatique (Inria) est un institut français de recherche en mathématiques et informatique, créé le 3 janvier 1967, ayant pour objectif la mise en réseau des compétences et talents de l'ensemble du dispositif de recherche français et international dans ses domaines de compétence. Il dispose de huit centres de recherches autonomes répartis sur tout le territoire français :
	
	\begin{itemize}
		\item Bordeaux - Sud-Ouest
		\item Grenoble - Rhône-Alpes
		\item Lille - Nord Europe
		\item Nancy - Grand Est
		\item Paris - Rocquencourt
		\item Rennes - Bretagne Atlantique
		\item Saclay - Île-de-France
		\item Sophia Antipolis - Méditerranée
	\end{itemize}

	\subsubsection{RMoD}
	RMoD est une équipe de recherche de l'Inria basée dans le centre de Lille - Nord Europe, et dirigée par Stéphane Ducasse. Elle a principalement deux objectifs : le refactoring de grands systèmes et la création de langages de programmation dynamiques et réflexifs.
	
	Dans l'optique de refactoring de grands systèmes, l'équipe propose des outils d'analyse pour comprendre et restructurer les logiciels. Cette partie est assurée par Synectique, une entreprise créée dans ce but.
	
	Dans l'optique de création de langages de programmation, RMoD développe le langage Pharo, un langage dynamique, réflexif et purement objet basé sur Smalltalk.