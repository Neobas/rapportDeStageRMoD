% Section 1 : Contexte du stage
	\section{Contexte du stage}
	\subsection{L'entreprise}
	\subsubsection{Inria}

	\subsubsection{RMoD}
	RMoD est une équipe de recherche en \textbf{génie logiciel} basée dans le centre Inria de Lille - Nord Europe et dirigée par Stéphane Ducasse. Son principal but est l'évolution des systèmes sur le long terme, effectuée sur 2 axes : l'évolution de systèmes existants (appelés également systèmes legacy) , et la création de nouveaux systèmes aisément améliorables.
	
	\subsection{Technologies employées}
	\paragraph{Pharo}
	Pharo est un langage de programmation open source créé en 2008 par l'équipe RMoD, issu d'un fork de Squeak. Il est basé sur une machine virtuelle (VM) écrite en grande partie en Pharo lui-même, ce qui lui donne la capacité d'être multiplateforme (Mac OS X, Windows, Linux, iOS, Android). Sa politique impose que ses contributeurs publient leur code sous licence MIT.
	Basé sur Smalltalk, Pharo en possède les principales caractéristiques:
	\begin{itemize}
		\item Le système est réflexif.
		\item Le typage est dynamique.
		\item Pharo est implémenté suivant le paradygme objet
		\item L'héritage est simple.
		\item La gestion de la mémoire se fait automatique au moyen d'un garbage collector (GC)
	\end{itemize}
	Détail intéressant : la syntaxe de Pharo tient sur une carte postale.\\
	L'un des principaux intérêts de Pharo est qu'il n'est pas nécessaire de recompiler tout le code en cas de modification. Il est par exemple possible de coder au sein du débugger : On écrit un test, on le lance, une erreur apparait (le message envoyé n'existe pas), on peut alors ouvrir le débugger, implémenter la méthode correspondante, compiler à la volée et reprendre le cours de l'exécution comme si rien ne s'était produit.
	Pharo est actuellement disponible dans sa version 6.0, sortie le 6 juin 2017.

	\subsubsection{git}
	Git est un logiciel libre de gestion de versions décentralisé créé par Linus Torvalds, et distribué sous la licence publique générale GNU (GPL-2.0). Il est particulièrement utile pour gérer les modifications apportées à un projet grâce à un système de commits et de branches, surtout si il est couplé à un service en ligne capable de le gérer (par exemple github, pour ne citer que le plus connu).
